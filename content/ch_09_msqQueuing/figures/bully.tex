% Created 2016-08-17 Wed 14:38
\documentclass[tikz]{standalone}

\usepackage{xparse}

\usepackage[utf8]{inputenc}
\usepackage[T1]{fontenc}
\usepackage{helvet}
\usepackage{../../templates/msc}

\renewcommand{\familydefault}{\sfdefault}

\tikzset{
every picture/.style={
line width=1pt
}}

\usepackage{tikz}
\author{Holger Karl}
\date{\today}
\title{}

\usepackage{marvosym}

\begin{document}

% state machine 

\newcommand{\numnodes}{7}
\newcommand{\diam}{3cm}
\pgfmathsetmacro{\scopesgrid}{2*\diam+1.5cm}

\newcommand{\leader}[1]{
  \node [star, draw, green, fill=green!60] at (#1*360/\numnodes:0.5cm + \diam) {}; 
}

\newcommand{\fail}[1]{
  \node [text=red] at (#1*360/\numnodes:0.5cm + \diam) {\Huge\Lightning}; 
}

\newcommand{\suspect}[1]{
  \node [text=blue] at (#1*360/\numnodes:0.5cm + \diam) {\Huge \textbf{?}}; 
}

\newcommand{\asknodes}[1]{
  \pgfmathsetmacro{\Start}{#1+1}
  \foreach \i in {\Start,...,\numnodes} {
    \draw [->, blue] (#1) to [bend left=20] (\i);
  }
}

\newcommand{\answernodes}[1]{
  \pgfmathsetmacro{\End}{#1-1}
  \foreach \i in {1,...,\End} {
    \draw [->, green] (#1) to [bend left=20] (\i);
  }
}


% \newcommand{\mygraph}[1]{% 
\NewDocumentCommand \mygraph {o} {
  \draw [dotted] (0,0) circle (\diam); 

  \foreach \i in {1,...,\numnodes} {
    \node [circle, draw, fill=white] (\i) at (\i*360/\numnodes:\diam) {\i};
  }

  \IfNoValueF{#1} {
    \node at (-\diam, \diam) {Step #1};
  }
}

\begin{tikzpicture}
  \mygraph
  \leader{7}
  \fail{6}
  \suspect{4}
  \asknodes{4}
  \answernodes{7}
\end{tikzpicture}

% regular process: 
\begin{tikzpicture}
  \begin{scope}
    \mygraph[1]
    \leader{7}
    \suspect{4}
  \end{scope}
  \begin{scope}[xshift=\scopesgrid]
    \mygraph[2]
    \leader{7}
    \suspect{4}
    \asknodes{4}
  \end{scope}
  \begin{scope}[yshift=-\scopesgrid]
    \mygraph[3]
    \leader{7}
    \suspect{4}
    \suspect{6}
    \asknodes{6}
    \answernodes{6}
  \end{scope}
  \begin{scope}[xshift=\scopesgrid, yshift=-\scopesgrid]
    \mygraph[4]
    \leader{7}
    \answernodes{7}
  \end{scope}
\end{tikzpicture}

% failed intermediate node 
\begin{tikzpicture}
  \begin{scope}
    \mygraph[1]
    \leader{7}
    \fail{6}
    \suspect{4}
  \end{scope}
  \begin{scope}[xshift=\scopesgrid]
    \mygraph[2]
    \leader{7}
    \fail{6}
    \suspect{4}
    \asknodes{4}
  \end{scope}
  \begin{scope}[xshift=0.5*\scopesgrid, yshift=-\scopesgrid]
    \mygraph[3]
    \leader{7}
    \fail{6}
    \answernodes{7}
  \end{scope}
\end{tikzpicture}


% failed leader 
\begin{tikzpicture}
  \begin{scope}
    \mygraph[1]
    \leader{7}
    \fail{6}
  \end{scope}
  \begin{scope}[xshift=\scopesgrid]
    \mygraph[2]
    \leader{7}
    \fail{6}
    \fail{7}
  \end{scope}
  \begin{scope}[xshift=2*\scopesgrid]
    \mygraph[3]
    \leader{7}
    \fail{6}
    \fail{7}
    \suspect{4}
  \end{scope}
  \begin{scope}[xshift=0*\scopesgrid, yshift=-\scopesgrid]
    \mygraph[4]
    \leader{7}
    \fail{6}
    \fail{7}
    \suspect{4}
    \asknodes{4}
  \end{scope}
  \begin{scope}[xshift=2/2*\scopesgrid, yshift=-\scopesgrid]
    \mygraph[5]
    \leader{7}
    \fail{6}
    \fail{7}
    \suspect{5}
    \asknodes{5}
    \answernodes{5}
  \end{scope}
  \begin{scope}[xshift=4/2*\scopesgrid, yshift=-\scopesgrid]
    \mygraph[6]
    \fail{6}
    \fail{7}
    \leader{5}
  \end{scope}
\end{tikzpicture}


\end{document}

